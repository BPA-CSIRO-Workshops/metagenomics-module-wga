% Define the module top matter
% This gets used to create the chapter title page
% NOTES:
%  * When multiple people have authored or contributed to the module, simply use a LaTeX line break
%    (a double-backslash: \\) at the end of each person.
%  * If you don't want this information shown on the module chapter page, simply remove the lines
%    within the \setModuleAuthors{} and \setModuleContributions{} environments
\setModuleTitle{Whole Genome Analysis}
\setModuleAuthors{%
  Hubert Denise, EBI Hinxton, UKs \mailto{hudenise@ebi.ac.uk} \\
}
\setModuleContributions{%
  BPA-CSIRO Trainers \mailto{agilbert@bioplatforms.com}%
}

% BEGIN: Module Title Page
% This simply uses the above information and creates a module chapter page
% NOTES:
%  * The chapter page will always appear on odd numbered page
\chapter{\moduleTitle}
\newpage
% END: Module Title Page


% BEGIN: KLOs
% Key Learning Outcomes (KLOs) are an important aspect of any learning/training. They provide
% valuable infomation about what the trainee will have learned, what they will be able to do or know
% abouti at the end of the module. Unlike objectives which are more trainer oriented, KOLs are
% focused on the learner.
% At the end of the module, the KLOs can be used to develop criteria for writing an assessment to
% see if the trainees knowledge/skills have improved as a result of the module.
% 
% Search online for information on how to write KLOs. e.g.
% http://www.teaching-learning.utas.edu.au/__data/assets/word_doc/0014/23333/Learning-outcomes-v9.1.doc
\section{Key Learning Outcomes}

After completing this module the trainee should be able to:
\begin{itemize}
  \item Understand the main approaches to perform metagenomics assembly
  \item Be able to perform assembly on your data and assess the quality of your assembly
\end{itemize}
% END KLOs

% BEGIN: Resources Used
% This section can be used to describe the tools and data used during the module. It helps to act as
% a future reference to the trainee
\section{Resources You'll be Using}
 
\subsection{Tools Used}
\begin{description}[style=multiline,labelindent=0cm,align=left,leftmargin=0.5cm]
  \item[Meta-Velvet]\hfill\\
  	\url{https://github.com/hacchy/MetaVelvet}
\end{description}

\section{Useful Links}
 
\begin{description}[style=multiline,labelindent=0cm,align=left,leftmargin=0.5cm]
  \item[Meta-Velvet]\hfill\\
    \url{http://metavelvet.dna.bio.keio.ac.jp/}
\end{description}

\newpage
% END: Resources Used

% BEGIN: Introduction
\section{Introduction}

% To make a paragraph appear as a "note" to the reader, simply wrap it in a "note" environment like
% this:
\begin{note}
Performing genomic assembly aims at generating a genome-length sequence using the sequence information obtained from short reads. In the case of metagenomics sample, the task is complicated by the number of different genomes present in the sample and the fact that their sequences are sometimes very similar to each other. There are two main approaches to perform de novo assembly (genomic or metagenomic): building a consensus and generating De Bruijn k-mer graph. 
\end{note}

The k-mers represent the nodes of the de Bruijn graph. Nodes are linked together if they overlap by k-1 nucleotides. Determining the correct k-mer is important. You can use tools such as Velvet Advisor: \url{http://dna.med.monash.edu.au/~torsten/velvet_advisor/}

Building a de Bruijn graph is computationally demanding but navigation through it to identify path (to generate contigs of continuous sequences) is quick and memory efficient. Ideally other information, biological or distance-based, would be used to help build the contigs.
% END: Introduction

% BEGIN: Assessing the quality of an assembly
\subsection{Assessing the quality of an assembly}

For genomic assembly, the accepted criterion of assembly quality is the number of contigs obtained: the lower this number, the longer the contigs and therefore the higher genome reconstitution.
This number is often expressed as N50, which is defined as the weighted median such that 50\% of the entire assembly is contained in contigs equal to or larger than this value. It is calculated by ranking the contigs by decreasing length and adding their size sequentially until 50\% of the total number of nucleotides is reached: the N50 is defined by the number of contigs included in this sum.
The N50 is also generally used for metagenomics
% END: Assessing the quality of an assembly

% BEGIN: Practical
\section{Practical}
The purpose of this exercise is to perform an assembly using Meta-Velvet and illustrate how k-mer choices influence the output quality.
The starting dataset will be a metagenomic dataset. To ensure better assembly, rather than using the raw reads, we will only assemble the sequences having passed the EMG QC steps.
Meta-Velvet is an extension of Velvet, a popular genomic assembler. Therefore to perform our assembly, we will first run Velvet and then Meta-Velvet using the Velvet output as input. Both programs are run from the command line. 

% BEGIN: Investigation of the input sequence file
\subsection{Investigation of the input sequence file}

\begin{steps}
Open a terminal window (Applications/Accessories/Terminal, a link is also provided on your desktop) and navigate to the “data” folder in “AssemblyTutorial” and look at the first lines of the sequence file. The file is in fasta format and contains sequences of at least 100 nt each
\begin{lstlisting}
cd ~/Desktop/AssemblyTutorial/data
head A7A_processed.fasta
# What is the total number of sequences?
grep -c ">" A7A_processed.fasta
# What is the total number of nucleotides?
~/AssemblyTutorial/stats A7A_processed.fasta
\end{lstlisting}
\end{steps}

The output indicates that the input file contains about 2.2 billion (2,164,714,530) nucleotides distributed among ~ 21 million (20,975,212) sequences. The average sequence length was 103.2 nucleotides. The sequence file also contains 12,942 “N” indicating that some sequences have ambiguous bases. In addition, the script displays the N50 to N100 values: 
N50 = 101, n = 10256045 for example means that a cumulated sum of, at least, half of the total nucleotides is reached after adding the length of 10,256,045 sequences and that the last sequence added had a length of 101 nucletotides.
% END: Investigation of the input sequence file

% BEGIN: Performing the assembly using Velvet and Meta-Velvet
\subsection{Performing the assembly using Velvet and Meta-Velvet}

Velvet and Meta-Velvet had already been installed on your computer. However they need to be configured by indicating the k-mer value and the number of read categories to use. We already have seen what k-mer is (reminder in page 5 above). The number of read categories is the maximum number of libraries of different insert lengths. As in our case the reads are all coming from the same library, we will use the default value (2).
The version of Velvet and MetaVelvet installed on the virtual machine you will be using as already been configured with k-mer = 59.

\begin{steps}
We will run the applications from the AssemblyTutorial folder. The software has been installed in your path so no need to copy/link these files:
\begin{lstlisting}
# first run velveth to generate the k-mers:
velveth A7A-59 59 -fasta data/A7A_processed.fasta 
\end{lstlisting}
\end{steps}

\begin{steps}
Then run velvetg to construct the de Bruijn graph. The "- exp_cov auto" parameter indicates to the software that the sequence coverage is considered uniform across the submitted set and that the expected coverage (i.e. number of reads per sequence) should be the median coverage value:
\begin{lstlisting}
velvetg A7A-59 –exp_cov auto
\end{lstlisting}
\end{steps}

\begin{steps}
Finally run meta-velvetg to generate the assembly output in the A7A-59 directory:
\begin{lstlisting}
meta-velvetg A7A-59 | tee logfile 
\end{lstlisting}
\end{steps}
% END: Performing the assembly using Velvet and Meta-Velvet

% BEGIN: Assessing the quality of the assembly 
\subsection{Assessing the quality of the assembly}
The main assembly output is a list of contigs provided as a fasta file. We will know look at these in more details. First we need to navigate to the output directory:

\begin{steps}
Finally run meta-velvetg to generate the assembly output in the A7A-59 directory. We can obtain the number of contigs by running the function grep to only count the lines containing the contig names (identified by “>”). Then run the stats script, seen earlier, to also obtain the N50 value:
\begin{lstlisting}
cd A7A-59
grep -c ">" meta-velvetg.contigs.fa
~/AssemblyTutorial/stats meta-velvetg.contigs.fa
\end{lstlisting}
\end{steps}

It shows that the sequences had been assembled in 9,182 contigs of an average length equal to about 1,230 nucleotides. The longest contigs contains 95,305 nucleotides. The N50 line indicates that half of the nucleotides are comprised in the first 275 longest contigs and that the 275th contigs is 9,145 nucleotides long.
Comparing these stats to the one obtained before assembly also reveal that the number of nucleotides involved in the assembly represents only slightly more than 0.5\% of the nucleotides submitted. This reflects, of course, the amount of overlapping sequences identified by Velvet and MetaVelvet. This also explains the reduction of the number of ambiguous base (N_count).

\begin{warning}
Changing the k-mer value can have a dramatic effect on the quality of the assembly. Reducing the k-mer to 31 for example yields the following statistics:
\begin{lstlisting}
sum = 15527668, n = 42343, ave = 366.71, largest = 93835
N50 = 1208, n = 2151
N60 = 649, n = 3903
N70 = 328, n = 7388
N80 = 193, n = 13589
N90 = 115, n = 24202
N100 = 61, n = 42343
N_count = 263
\end{lstlisting}

The number of contigs is almost 5 times larger than with k-mer equal 59. However, increasing the k-mer alone does not systematically lead to better stats. A k-mer of 63 produces an assembly with higher number of contigs (Note that the N50 value is also increased):

\begin{lstlisting}
sum = 11579884, n = 10062, ave = 1150.85, largest = 80291
N50 = 7440, n = 365
N60 = 4524, n = 563
N70 = 2251, n = 921
N80 = 827, n = 1795
N90 = 326, n = 4155
N100 = 125, n = 10062
N_count = 1850
\end{lstlisting}

Without extra information, it could be challenging, using the N50 parameter, to judge the quality of a metagenomics assembly. We could use blast or other tools to infer taxonomy to different sections of the contigs: obtaining similar affiliation for all fragments originating from the same contigs would be indicative of a good assembly.

\end{warning}

%END: Assessing the quality of the assembly 
%END: Practical
